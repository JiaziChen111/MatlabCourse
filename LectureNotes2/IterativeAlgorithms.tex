\documentclass[english,xcolor=dvipsnames]{beamer}
% load package with ``framed'' and ``numbered'' option.
\usepackage[autolinebreaks,useliterate]{mcode}
\usepackage[orientation=landscape,size=custom,width=16,height=9,scale=0.48,debug]{beamerposter}
\usepackage[T1]{fontenc}
\usepackage[latin9]{inputenc}
\usepackage{amsthm}
\usepackage{amsmath}
\usepackage{amssymb}
\usepackage{bookmark}
\usepackage{verbatim}
\usepackage{graphics,graphicx}
\usepackage{pstricks,pst-node,pst-tree}
\usefonttheme{serif}
\usepackage{palatino}
%\usepackage[margin=.5cm]{geometry}

\definecolor{dgreen}{rgb}{0.,0.6,0.}
\definecolor{forest}{RGB}{34.,139.,34.}
\definecolor{byublue}{RGB}{0.,30.,76.}
\definecolor{dukeblue}{RGB}{0.,0.,156.}
%\usetheme{Ilmenau}
\usetheme{Warsaw}
\usecolortheme[named=dukeblue]{structure}
%\usecolortheme[named=RawSienna]{structure}
%\usecolortheme[named=byublue]{structure}
\setbeamertemplate{navigation symbols}{}
\setbeamertemplate{footline}{}
\setbeamercovered{transparent}

%%%%%%%%%%%%%%%%%%%%%%%%%%%%%%%%%%%%%%%%%%%%%%%%
% NOTE: With Ilmenau style, to get the bullets %
% looking right, do one section and one sub-   %
% section for each set of bullets              %
%%%%%%%%%%%%%%%%%%%%%%%%%%%%%%%%%%%%%%%%%%%%%%%%

\begin{document}
\begin{frame}
\title{Iterative Estimation Algorithms}
\author{
	Tyler Ransom\\
	\emph{Duke University}\\
%    \today \\
%    \vspace{10cm}
}
\titlepage
\end{frame}

\begin{frame}{Outline}
\begin{itemize}
	\item Introduce various commonly used estimation algorithms in empirical micro
\end{itemize}
\end{frame}

\begin{frame}{Multistage Estimation Algorithms}
\begin{itemize}
	\item In advanced econometrics, complex algorithms abound
	\item These algorithms pare down the dimensionality of a problem so that computation is much easier
	\item Typically, these algorithms rely on certain theoretical assumptions, so they may not be generalizable to all types of problems
\end{itemize}
\end{frame}

\begin{frame}{Multistage Estimation Algorithms}
\begin{itemize}
	\item Another property of these algorithms is that they typically induce estimation error between the multiple steps of estimation
	\item Hence, inference on the final parameter estimates may require bootstrapping in order to recover (an approximation to) the true variance matrix of the parameter estimates
	\item At the very least, some sort of correction to the standard errors is required
\end{itemize}
\end{frame}

\begin{frame}{Examples of Multistage Estimation Algorithms}
\begin{itemize}
	\item Expectation Maximization (EM) algorithm
	\item Berry, Levinsohn and Pakes  (BLP) method for demand estimation
	\item Bajari, Benkard and Levin (BBL) method for solving dynamic games
	\item Iterative algorithm for multi-dimensional fixed effects
	\item We'll go through each of these in more detail
\end{itemize}
\end{frame}

\begin{frame}{EM Algorithm}
\begin{itemize}
	\item A famous algorithm for estimating models with missing data
	\item Can be applied to finite mixture unobserved heterogeneity models
	\item ``E'' step: given parameters $\theta$, estimate the probability that each person is of finite mixture type $R$
	\item ``M'' step: given type probabilities, estimate the parameters $\theta$
	\item Iterate until parameters $\theta$ converge
	\item You will see this again if you take Peter Arcidiacono's dynamic discrete choice class (which I highly recommend)
\end{itemize}
\end{frame}

\begin{frame}{BLP method}
\begin{itemize}
	\item This is a famous method in IO for estimating demand for products, where the demand (market share) is a function of product characteristics interacted with consumer characteristics
	\item The estimation method is quite complicated.
	\item Involves a contraction mapping over consumer mean utilities
	\item This contraction mapping is nested within a GMM algorithm
	\item This is called a ``Contraction Mapping Fixed Point'' algorithm
	\item Allows researcher to recover random coefficients for the consumers
\end{itemize}
\end{frame}

\begin{frame}{BBL method}
\begin{itemize}
	\item Estimation algorithm for dynamic games where players are assumed to play Markov Perfect Equilibria
	\item Step 1: estimate the policy functions and law of motion for state variables
	\item Step 2: given the Step 1 estimates, estimate the structural parameters given the FOCs that are consistent with MPE
	\item Iterate until structural parameters converge
\end{itemize}
\end{frame}

\begin{frame}{Multidimensional Fixed Effects}
Suppose we have an outcome equation 
\[
y_{ijt} = X_{ijt}\beta + \theta_{i} + \theta_{j} + \theta_{i} \theta_{j} + \varepsilon_{ijt}
\]
where $\theta_{i}$ and $\theta_{j}$ are fixed effects for two different agents e.g. worker and firm, student and teacher, individual and village
\end{frame}

\begin{frame}{Multidimensional Fixed Effects}
\begin{itemize}
	\item Arcidiacono et al. (2011) outlines a simple estimation routine that will allow the researcher to recover the fixed effects without having to use LSDV
	\item Step 1: Given values for $\theta_{i}$, estimate $\theta_{j}$ and $\beta$ by OLS
	\item Step 2: Given the estimates for $\theta_{j}$ and $\beta$, estimate $\theta_{i}$ by OLS
	\item Iterate until all parameters converge
\end{itemize}
\end{frame}

\end{document}
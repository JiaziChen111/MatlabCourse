\documentclass[english,xcolor=dvipsnames]{beamer}
% load package with ``framed'' and ``numbered'' option.
\usepackage[autolinebreaks,useliterate]{mcode}
\usepackage[orientation=landscape,size=custom,width=16,height=9,scale=0.48,debug]{beamerposter}
\usepackage[T1]{fontenc}
\usepackage[latin9]{inputenc}
\usepackage{amsthm}
\usepackage{amsmath}
\usepackage{amssymb}
\usepackage{bookmark}
\usepackage{verbatim}
\usepackage{graphics,graphicx}
\usepackage{pstricks,pst-node,pst-tree}
\usefonttheme{serif}
\usepackage{palatino}
%\usepackage[margin=.5cm]{geometry}

\definecolor{dgreen}{rgb}{0.,0.6,0.}
\definecolor{forest}{RGB}{34.,139.,34.}
\definecolor{byublue}{RGB}{0.,30.,76.}
\definecolor{dukeblue}{RGB}{0.,0.,156.}
%\usetheme{Ilmenau}
\usetheme{Warsaw}
\usecolortheme[named=dukeblue]{structure}
%\usecolortheme[named=RawSienna]{structure}
%\usecolortheme[named=byublue]{structure}
\setbeamertemplate{navigation symbols}{}
\setbeamertemplate{footline}{}
\setbeamercovered{transparent}

%%%%%%%%%%%%%%%%%%%%%%%%%%%%%%%%%%%%%%%%%%%%%%%%
% NOTE: With Ilmenau style, to get the bullets %
% looking right, do one section and one sub-   %
% section for each set of bullets              %
%%%%%%%%%%%%%%%%%%%%%%%%%%%%%%%%%%%%%%%%%%%%%%%%

\begin{document}
\begin{frame}

\title{Duration Analysis \& Count Data%
\thanks{Based on Wooldridge (2003) Chapters 19-20%
}}


\author{Tyler Ransom}


\institute{Duke University}

\titlepage

\end{frame}


\begin{frame}{Introduction}
\begin{itemize}
\item Today we'll discuss how to estimate a model where the dependent variable
takes on two irregular types: a time interval, and a positive integer
\item When the dependent variable is a time interval, we use duration analysis
to recover the parameters of interest
\item When the dependent variable is a positive integer, we use count data
methods to estimate the parameters
\end{itemize}

\end{frame}


\begin{frame}{When $y$ is a time interval}

When our dependent variable $y$ is a time interval, we use duration
analysis to estimate the model, which helps us understand the cause(s)
of the dependent variable. 
\begin{itemize}
\item Examples include:
\item weeks unemployed until finding a job
\item months married until divorce
\item days until arrest after incarceration
\item survey rounds until attrition
\item weeks pregnant until delivery
\end{itemize}

\end{frame}


\begin{frame}{Background}
\begin{itemize}
\item For now, we assume that time is continuous
\item The dependent variable is known as the {}``\textbf{initial state}''
(e.g. marriage is the {}``initial state'' in the divorce example.
Someone exits the initial state when they become divorced)
\item We also assume that there is no re-entry into the initial state

\begin{itemize}
\item Therefore, we only consider spells until the first exit
\item Thus, $y$ would be comprised of first marriages (for the divorce
example) or first-time offenders (in the arrest example)
\end{itemize}
\item For now, we also assume there are no covariates, i.e. we only look
at unconditional time to exit
\end{itemize}

\end{frame}


\begin{frame}{Hazard function}

In order to better understand the econometrics of duration models,
let's introduce some notation:
\begin{itemize}
\item Let $T$ be a random variable with outcomes $t$ and continuous distribution
$F$ with support $\left(0,\infty\right)$
\item $F\left(t\right)=\Pr\left(T\leq t\right),\,\, t\geq0$
\item The \textbf{survivor function} is the probability of surviving past
$t$ and is denoted $1-F\left(t\right)$
\item The \textbf{hazard function} $\lambda\left(t\right)$ is the instantaneous
rate of exiting the initial state per unit of time and is defined
by
\[
\lambda\left(t\right)=\lim_{h\downarrow0}\frac{\Pr\left(t\leq T\leq t+h\,\,\vert\,\, T\geq t\right)}{h}
\]

\end{itemize}

\end{frame}


\begin{frame}{Hazard function}

Some examples of the hazard function:
\begin{itemize}
\item If $T$ is unemployment length (in weeks) then $\lambda\left(20\right)$
is approximately the probability of accepting a job offer between
weeks 20 and 21 (conditional on not having accepted a job before then)
\item If $T$ is length of first marriage (in months) then $\lambda\left(84\right)$
is the approximate probability of becoming divorced between months
84 and 85
\end{itemize}

\end{frame}


\begin{frame}{Hazard function}
\begin{itemize}
\item You can use math to crank out the following using the definition of
the hazard function given previously:
\[
\lambda\left(t\right)=\frac{f\left(t\right)}{1-F\left(t\right)}.
\]

\item The simplest form of a hazard function is $\lambda\left(t\right)=\gamma$
(some constant)
\item Thus we have
\begin{eqnarray*}
\gamma & = & \frac{f\left(t\right)}{1-F\left(t\right)}\\
f\left(t\right) & = & \gamma\left(1-F\left(t\right)\right)
\end{eqnarray*}
which is an ordinary differential equation. The solution is
\[
F\left(t\right)=1-e^{-\gamma t}
\]
which is the CDF of the exponential distribution.
\end{itemize}

\end{frame}


\begin{frame}{Duration dependence}
\begin{itemize}
\item The exponential distribution has the property of \textbf{memorylessness}
\item This means there is no \textbf{duration dependence} (i.e. the probability
of exiting the initial state is the same regardless of how long you've
been in the initial state)
\item In practice, this is usually not a realistic assumption
\item We usually want distributions that exhibit duration dependence, meaning
$\lambda\left(t\right)$ is actually a function of $t$ (not a constant)
\item Common distributions that exhibit duration dependence but aren't too
hairy to work with:

\begin{itemize}
\item Weibull distribution
\item Log-logistic distribution
\end{itemize}
\end{itemize}

\end{frame}


\begin{frame}{Introducing covariates}

If we want to study the hazard rate conditional on a matrix of covariates
$X$, for $X$'s that do not vary over time (i.e. are measured at
the start of the spell):
\[
\lambda\left(t;X\right)=\frac{f\left(t\,\vert\, X\right)}{1-F\left(t\,\vert\, X\right)}
\]

\begin{itemize}
\item This leads us to the \textbf{proportional hazards model}:
\[
\lambda\left(t;X\right)=\kappa\left(X\right)\lambda_{0}\left(t\right)
\]
where we assume that the hazard function is separable between $X$
and $t$. In the case, we call $\lambda_{0}$ the \textbf{baseline
hazard}
\end{itemize}

\end{frame}


\begin{frame}{Functional form}
\begin{itemize}
\item In a proportional hazards model, it is common to assume
\[
\kappa\left(\cdot\right)=e^{X\beta}
\]

\item Plugging this back into the conditional hazard function gives us
\begin{equation}
\lambda\left(t;X\right)=\lambda_{0}\left(t\right)\exp\left(X\beta\right)
\end{equation}

\item Simplifying, we have
\[
\ln\left(\lambda\left(t;X\right)\right)=X\beta+\ln\left(\lambda_{0}\left(t\right)\right)
\]

\item Remember, $X$ cannot be time-varying!
\end{itemize}

\end{frame}


\begin{frame}{Time-varying covariates}
\begin{itemize}
\item We can introduce time-varying covariates $X\left(t\right)$, but we
need to be a lot more careful
\item In particular, the $X$'s must be strictly exogenous (meaning $X\left(t\right)$
needs to have a well defined path once an agent leaves the initial
state)
\item If these stricter assumptions are met, we can write the proportional
hazard as
\[
\lambda\left(t;X\left(t\right)\right)=\kappa\left(X\left(t\right)\right)\lambda_{0}\left(t\right)
\]
with
\[
\kappa\left(\cdot\right)=\exp\left(X\left(t\right)\beta\right)
\]

\item If we assume $T\sim$log-logistic then
\[
\lambda\left(t;X\left(t\right)\right)=\frac{\alpha t^{\alpha-1}\exp\left(X\left(t\right)\beta\right)}{1+t^{\alpha}\exp\left(X\left(t\right)\beta\right)}
\]
and we can estimate $\alpha,\beta$ by MLE
\end{itemize}

\end{frame}


\begin{frame}{Estimation}

Estimation of duration data depends on the sampling scheme. Let's
suppose $T$ is the duration of unemployment for a male in the year
1998. To get data, we can sample in one of two ways:
\begin{enumerate}
\item flow sampling: Randomly sample men who became unemployed in 1998
\item stock sampling: Randomly sample men who were unemployed in the last
week of 1998\end{enumerate}
\begin{itemize}
\item In both cases, we need to correct for right censoring, because there
will be people who are still in the initial state at the time of sampling
\item In the case of stock sampling, we also need to correct for left censoring
\item Unobserved heterogeneity is much more difficult to correct for in
duration analysis, so we won't cover it today
\end{itemize}

\end{frame}


\begin{frame}{Estimation of expected duration}

Suppose we weren't interested in estimating the hazard function, just
the effect of $X$ on the expected duration $\mathbb{E}\left[t\right]$. 
\begin{itemize}
\item If there were no censoring, we could just run OLS
\[
\ln\left(t^{\ast}\right)=X\delta+e
\]
where $t^{\ast}$ is a vector of actual duration times for staying
in the initial state
\item If we assume $T\sim$Weibull, we can still estimate the expected duration
using OLS, but 
\[
\delta_{j}=\frac{-\beta_{j}}{\alpha}
\]
where $\delta_{j}$ is the effect of $X_{j}$ (the $j$th column of
$X$) on expected duration and $\alpha$ is a parameter of the Weibull
distribution
\item If there is right censoring, need to do standard tobit to estimate
$\delta$
\end{itemize}

\end{frame}


\begin{frame}{Estimation of the hazard function}

When estimating parameters of the hazard function (e.g. equation 1)
with flow sampling, need to do maximum likelihood and pay attention
to truncation. The likelihood function looks like
\begin{eqnarray*}
\mathcal{L}\left(t_{i};X_{i},\theta\right) & = & \prod_{i}f\left(t_{i}\,\vert\, X_{i};\theta\right)^{d_{i}}\left[1-F\left(t_{i}\,\vert\, X_{i};\theta\right)\right]^{1-d_{i}},\\
 &  & d_{i}=1\left[\mbox{uncensored}\right]
\end{eqnarray*}



\end{frame}


\begin{frame}{Likelihood for stock sampling}
\begin{itemize}
\item With stock sampling, we need to introduce some more notation:
\item $a_{i}$ is the starting time for person $i$
\item $b$ is the maximum length of the spell (e.g. 52 weeks for the {}``unemployed
in year 1998'' example)
\item The likelihood function is then
\[
\mathcal{L}\left(t_{i};X_{i},\theta\right)=\prod_{i}\frac{f\left(t_{i}\,\vert\, X_{i};\theta\right)^{d_{i}}\left[1-F\left(t_{i}\,\vert\, X_{i};\theta\right)\right]^{1-d_{i}}}{1-F\left(b-a_{i}\,\vert\, X_{i};\theta\right)}
\]

\end{itemize}

\end{frame}


\begin{frame}{Proportional hazard models in Matlab}
\begin{itemize}
\item The Matlab command for estimating a Cox proportional hazard model
(the most common type of duration analysis) is \texttt{coxphfit}
\item The full syntax is
\item \texttt{{[}b,logl,H,stats{]} = coxphfit(X,y,'name',value)}
\item \texttt{H} is a two-column matrix that contains $y$ values in the
first column and cumulative hazard values in the second
\item \texttt{'name'} is how you invoke optimization options
\end{itemize}

\end{frame}


\begin{frame}{Grouped data}

The assumption that time is continuous can be a hard pill to swallow,
especially when the data is measured on a weekly or monthly basis.
For example:
\begin{itemize}
\item We only have weekly data on arrests
\item but arrests could have happened any time in that week
\item We need to account for this discreteness in our estimation
\end{itemize}

\end{frame}


\begin{frame}{Estimation of grouped data}

We observe $y_{m},c_{m},X$ where $m$ indicates the time range (e.g.
$y_{m}=1\left[\mbox{exited\, initial\, state\, in\, week\,\ensuremath{m}\, of\,1998}\right]$,
$c_{m}=1\left[\mbox{censored\, in\, week\,}m\right]$, and $X$ is
a matrix of time-invariant covariates
\begin{itemize}
\item If we assume a \textbf{piecewise-constant proportional hazard}, then
\[
\lambda\left(t;X,\theta\right)=\kappa\left(X\beta\right)\lambda_{m},\, a_{m-1}\leq t<a_{m}
\]
and we get the likelihood function
\[
\mathcal{L}\left(X;\theta\right)=\prod_{i=1}^{N}\left[\prod_{h=1}^{m_{i}-1}\alpha_{h}\left(X_{i};\theta\right)\right]\left[1-\alpha_{m_{i}}\left(X_{i};\theta\right)\right]^{d_{i}}
\]
where
\[
\alpha_{m}=\lambda_{m}\left(a_{m}-a_{m-1}\right)\exp\left(-\exp\left(X\beta\right)\right)
\]
and $a_{m}$ is a time period (e.g. one week); $d_{i}$ is the same
as before
\end{itemize}

\end{frame}


\begin{frame}{Kaplan-Meier estimator}

One of the most famous estimators in duration analysis (used frequently
in the finance literature) is called the \textbf{Kaplan-Meier estimator
of the survivor function}.
\begin{itemize}
\item No covariates
\item Want to estimate $\lambda_{m}$ with grouped data
\[
\hat{s}\left(a_{m}\right)=\prod_{r=1}^{m}\frac{N_{r}-E_{r}}{N_{r}}
\]

\item where $N_{r}$ is the number of people who have neither left the initial
state nor been censored at $a_{r-1}$\\
$E_{r}$ is the number of people leaving the initial state between
$a_{r-1}$ and $a_{r}$
\end{itemize}

\end{frame}


\begin{frame}{Time-varying covariates in grouped data}

If we want to estimate grouped data with time-varying covariates,
then the covariates $X_{m}$ must not change within each $a_{m}$
\begin{itemize}
\item If this is the case, then we can rewrite the likelihood function from
before, but add an $m$ subscript to the $X$ matrix:
\[
\mathcal{L}\left(X;\theta\right)=\mathcal{L}\left(X;\theta\right)=\prod_{i=1}^{N}\left[\prod_{h=1}^{m_{i}-1}\alpha_{h}\left(X_{im};\theta\right)\right]\left[1-\alpha_{m_{i}}\left(X_{im};\theta\right)\right]^{d_{i}}
\]
where
\[
\alpha_{m}=\lambda_{m}\left(a_{m}-a_{m-1}\right)\exp\left(-\exp\left(X_{m}\beta\right)\right)
\]

\end{itemize}

\end{frame}


\begin{frame}{Competing risks models}

If we want to generalize our duration analysis to having more than
one exit state, this is called a \textbf{competing risks model}
\begin{itemize}
\item In our employment example, we considered employment as the {}``exit
state''
\item We could consider a model in which the individual can move either
from unemployment to employment, or from unemployment to out of the
labor force
\item In a competing risks model, each of the {}``risks'' is assumed to
be independent of one another
\item The econometrics get a lot more hairy, so we won't cover it today
\end{itemize}

\end{frame}


\begin{frame}{Models with count data}

\textbf{Count data} is a set of data where the dependent variable
takes on values that are nonnegative integers. In this case, $y$
is called a \textbf{count variable}. Examples include
\begin{itemize}
\item Number of cigarettes smoked in a day
\item Number of plane crashes experienced by an airline over a certain period
of time
\item Number of days spent in a hospital last year
\item Number of times visited a recreational site
\end{itemize}

\end{frame}


\begin{frame}{Count data}

Just like with binary models (where $y$ only takes on values in $\left\{ 0,1\right\} $),
we need to estimate the model differently
\begin{itemize}
\item The main reason is that if we performed OLS on $y=X\beta+\varepsilon$,
we would find that $\hat{y}$ would be negative for some values of
$X$

\begin{itemize}
\item This is the same reason we do a logit or probit model for binary data
\end{itemize}
\item In order to ensure that $\hat{y}$ is always positive, we make our
favorite transformation: $\exp\left(X\beta\right)$
\item We also need to assume a better distribution for $\varepsilon$
\end{itemize}

\end{frame}


\begin{frame}{Poisson regression}

Let's assume $y\vert X\sim$Poisson with mean parameter $\mu=\exp\left(X\beta\right)$
\begin{itemize}
\item The pdf of this distribution is
\[
f\left(y\vert X\right)=\frac{\exp\left(-\mu\left(X\beta\right)\right)\left[\mu\left(X\beta\right)\right]^{y}}{y!}
\]

\item Our quasi MLE (QMLE) likelihood function is then
\[
\ell=\sum_{i}y_{i}X_{i}\beta-\exp\left(X_{i}\beta\right)-\ln\left(y!\right)
\]

\end{itemize}

\end{frame}


\begin{frame}{Interpretation}

Interpretation of Poisson regression coefficients is unlike typical
regression models:
\[
\beta_{j}=\frac{\partial\ln\left(\mathbb{E}\left[y\vert X\right]\right)}{\partial X_{j}}
\]
Turning this into a differential and rearranging terms, we get
\[
\Delta\mathbb{E}\left[y\vert X\right]\approx100\beta_{j}\Delta X_{j}
\]
Thus $\beta_{j}$ is a semielasticity


\end{frame}


\begin{frame}{Other count distributions}

Besides the Poisson distribution, we can also use the following distributions
to fit our count data:
\begin{itemize}
\item Negative binomial
\item Binomial
\item Exponential
\item Fractional logit
\end{itemize}
See Wooldridge for more information regarding these. Also be aware
that inference is slightly more complicated with count models.


\end{frame}


\begin{frame}{Conclusion}
\begin{itemize}
\item Today we discussed how to estimate models where the dependent variable
takes a specific form that is different than what we're used to
\item You will likely come across duration analysis in the future
\item You may not come across count data, but if you do, you know where
to look for more information
\end{itemize}

\end{frame}
\end{document}

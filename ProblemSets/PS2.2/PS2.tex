%% LyX 2.0.0 created this file.  For more info, see http://www.lyx.org/.
%% Do not edit unless you really know what you are doing.
\documentclass[12pt,english]{article}
\usepackage{mathptmx}
\usepackage[T1]{fontenc}
\usepackage[latin9]{inputenc}
\usepackage{geometry}
\geometry{verbose,tmargin=1in,bmargin=1in,lmargin=1in,rmargin=1in}
\usepackage{babel}
\usepackage{amsmath}
\usepackage[unicode=true,pdfusetitle,
 bookmarks=true,bookmarksnumbered=false,bookmarksopen=false,
 breaklinks=false,pdfborder={0 0 0},backref=false,colorlinks=false]
 {hyperref}
\usepackage{breakurl}

\makeatletter
%%%%%%%%%%%%%%%%%%%%%%%%%%%%%% User specified LaTeX commands.
\date{}

\makeatother

\begin{document}

\title{Problem Set 2}

\maketitle
Directions: Answer all questions. Clearly label all answers. Show
all of your code. Compute standard errors for all parameter estimates.
Turn in the following to me via your dropbox (in a folder labeled
'MatlabPS2.2') in Sakai by 11:59 p.m. on Thursday, August 2, 2012: 
\begin{itemize}
\item m-file(s)
\item .sh shell script file that executes your m-file(s)%
\footnote{Name your job something other than {}``matsub,'' and have it send
you an email upon completion.%
}
\item a log file (from off the cluster)
\item matsub.oXXXXX file
\item pdf version of your writeup with its \LaTeX source code
\end{itemize}
Put the names of all group members at the top of your writeup (each
student must turn in his/her own materials).
\begin{enumerate}
\item Gradient of Multinomial Logit.

\begin{enumerate}
\item Derive the gradient of the multinomial logit log likelihood function,
i.e. find $\partial\ell/\partial\beta_{j}$ where 
\begin{eqnarray*}
\mathcal{L}\left(X_{it};\beta\right) & = & \prod_{i}\prod_{j}\prod_{t}P_{ijt}^{d_{ijt}}\\
\ln\left(\mathcal{L}\left(X_{it};\beta\right)\right) & = & \ln\left(\prod_{i}\prod_{j}\prod_{t}P_{ijt}^{d_{ijt}}\right)\\
\ell\left(X_{it};\beta\right) & = & \sum_{i}\sum_{j}\sum_{t}d_{ijt}\ln\left(P_{ijt}\right)\\
 & = & \sum_{i}\sum_{j}\sum_{t}d_{ijt}\ln\left(\frac{\exp\left(X_{ijt}\beta_{j}\right)}{\sum_{k}\exp\left(X_{ikt}\beta_{k}\right)}\right).
\end{eqnarray*}
Express this derivative in matrix form. For more help, see p. 63 of
Train%
\footnote{Available for free online at \url{http://elsa.berkeley.edu/books/choice2.html}.
Note that Train expresses his formula in terms of $\partial\ell/\partial\beta$,
not $\partial\ell/\partial\beta_{j}$.%
}.
\end{enumerate}
\item Estimation of Multinomial Logit with and without gradient. Using the
dataset \texttt{nlsy97m.mat}, consider the following model (same as
Problem Set 1)
\begin{eqnarray}
d_{ijt} & = & \beta_{1j}+\beta_{2j}male_{i}+\beta_{3j}AFQT_{i}+\beta_{4j}Mhgc_{i}+\beta_{5j}hgc_{it}+\beta_{6j}exper_{it}+\\
 &  & \beta_{7j}Diploma_{it}+\beta_{8j}AA_{it}+\beta_{9j}BA_{it}+\varepsilon_{ijt}\nonumber 
\end{eqnarray}
where $\varepsilon_{ijt}\overset{iid}{\sim}$Type I Extreme Value,
and $d_{ijt}$ is an indicator that $choice_{it}=j$ for individual
$i$ in time $t$. $choice_{it}$ now takes on values $1,\ldots,6$
according to the following fashion:

\begin{itemize}
\item school only
\item work while attending school
\item work part-time (no school)
\item work full-time (no school)
\item active military duty
\item all other activities\end{itemize}
\begin{enumerate}
\item Tabulate the choice set.
\item Estimate $\beta$ using \texttt{fminunc} and random starting values%
\footnote{As usual, set the seed to 1234.%
}. Normalize $\beta_{6}$ to be zero (i.e. use {}``other activities''
as the reference group). Report $\hat{\beta}$ and its standard errors,
along with the log likelihood value at convergence and number of iterations
to convergence.
\item Now estimate $\beta$ using \texttt{fminunc} and random starting values,
but this time supply \texttt{fminunc} with the gradient you computed
in part (a) of question 1.%
\footnote{Remember that you can use the \texttt{'DerivativeCheck','on'} option
in \texttt{fminunc} to compare your analytical derivative with Matlab's
numerical approximation.%
} What difference does the user-supplied gradient make in terms of
computational time as well as estimate precision?
\end{enumerate}
\item Constrained optimization within \texttt{fminunc}. Using the same dataset
as in question 2, estimate the following model:
\begin{eqnarray*}
\ln\left(wage_{it}\right) & = & \gamma_{1}+\gamma_{2}male_{i}+\gamma_{3}AFQT_{i}+\gamma_{4}Mhgc_{i}+\gamma_{5}hgc_{it}+\gamma_{6}exper_{it}+\\
 &  & \gamma_{7}Diploma_{it}+\gamma_{8}AA_{it}+\gamma_{9}BA_{it}+\gamma_{10}school\&work_{it}+\gamma_{11}workPT_{it}+\varepsilon_{it}
\end{eqnarray*}
where $\varepsilon_{it}\overset{iid}{\sim}N\left(0,\sigma^{2}\right)$.
$school\&work_{it}$ is a dummy for whether or not the individual
$i$ chose school and work (choice 2) in time $t$, and $workPT_{it}$
is a dummy for whether or not individual $i$ chose work part-time
(choice 3) in time $t$.

\begin{enumerate}
\item Estimate $\hat{\gamma}$ using \texttt{fminunc} and imposing the following
constraints within the optimization%
\footnote{Hint: The following constraint functions are compatible with these
constraints: $\exp(x)$, $-x^{2}$, $.15*\tanh(x)+.05$%
}: 

\begin{itemize}
\item $\hat{\sigma}>0$
\item $\hat{\gamma}_{10}<0$
\item $-.1<\hat{\gamma}_{2}<.2$
\end{itemize}

Be sure to report the delta-method-corrected standard errors with
your point estimates.

\end{enumerate}
\item Estimation of normal MLE with and without gradient and hessian.%
\footnote{For this problem, start all optimization at the same starting values
(one specific draw from a $U[0,1]$ distribution).%
}

\begin{enumerate}
\item Estimate the same model as in question 3, but this time supply the
gradient of the log likelihood to \texttt{fminunc}. Compare your results
with and without the user-provided gradient. Report coefficient estimates,
log likelihood values, iteration counts, and standard errors of your
coefficient estimates. Comment on the computational savings of a user-provided
gradient.
\item Repeat (a), but this time supply both the gradient and the hessian.%
\footnote{Note: Matlab does not have an analogous option to \texttt{'DerivativeCheck'}
for the user-supplied hessian. However, there is a user-written script
called \texttt{hessian} (the full package name is {}``Adaptive Robust
Numerical Differentiation'' which you can find on the Matlab file
exchange), which will numerically calculate the hessian of an objective
function. This is useful as a way to check that your differentiation
is correct.%
} For help on what this looks like, consult Hayashi (pp. 49-51) or
Greene (pp. 518-519). Compare your results with and without the user-provided
hessian. Report coefficient estimates, log likelihood values, iteration
counts, and standard errors of your coefficient estimates. Comment
on the computational savings of a user-provided hessian.\end{enumerate}
\end{enumerate}

\end{document}

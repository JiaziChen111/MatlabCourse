\documentclass[english,xcolor=dvipsnames]{beamer}
% load package with ``framed'' and ``numbered'' option.
\usepackage[numbered,framed,autolinebreaks,useliterate]{mcode}
\usepackage[orientation=landscape,size=custom,width=16,height=9,scale=0.48,debug]{beamerposter}
\usepackage[T1]{fontenc}
\usepackage[latin9]{inputenc}
\usepackage{amsthm}
\usepackage{amsmath}
\usepackage{amssymb}
\usepackage{bookmark}
\usepackage{graphics,graphicx}
\usepackage{pstricks,pst-node,pst-tree}
\usefonttheme{serif}
\usepackage{palatino}
\usepackage{tikz}
\usetikzlibrary{shapes,arrows}
\usetikzlibrary{positioning}
%\usepackage[margin=.5cm]{geometry}

\definecolor{dgreen}{rgb}{0.,0.6,0.}
\definecolor{forest}{RGB}{34.,139.,34.}
\definecolor{byublue}{RGB}{0.,30.,76.}
\definecolor{dukeblue}{RGB}{0.,0.,156.}
%\usetheme{Ilmenau}
\usetheme{Warsaw}
\usecolortheme[named=dukeblue]{structure}
%\usecolortheme[named=RawSienna]{structure}
%\usecolortheme[named=byublue]{structure}
\setbeamertemplate{navigation symbols}{}
\setbeamertemplate{footline}{}
\setbeamercovered{transparent}

\newcommand{\be}{\begin{enumerate}}
\newcommand{\ee}{\end{enumerate}}
\newcommand{\bq}{\begin{quote}}
\newcommand{\eq}{\end{quote}}
\newcommand{\bd}{\begin{description}}
\newcommand{\ed}{\end{description}}
\newcommand{\bi}{\begin{itemize}}
\newcommand{\ei}{\end{itemize}}

\begin{document}
\begin{frame}
\title{Graphics and Exportation in Matlab}
\author{
	Tyler Ransom\\
	\emph{Duke University}\\
%    \today \\
%    \vspace{10cm}
}
\titlepage
\end{frame}

\begin{frame}
\frametitle{Exporting Estimation Results}
   \bi 
   \item Exporting estimation results from Matlab to Excel or LaTeX is unfortunately not so straightforward
   \item This is mostly due to Matlab's inability to simultaneously store numbers and characters in a matrix
   \item We'll go through the pros and cons to each method
   \ei
\end{frame}

\begin{frame}
\frametitle{Matlab to Excel (method 1)}
   \bi 
   \item Create a matrix that concatenates results
   \item e.g. \mcode{results = [beta std_err beta./std_err];}
   \item After estimation has completed, open the matrix ``results'' in the workspace, and copy and paste it into Excel
   \item Pro: Very user friendly and quick
   \item Con: Not reproducible, so if you need to re-run the estimation, you also will need to re-copy and paste
   \ei
\end{frame}

\begin{frame}
\frametitle{Matlab to Excel (method 2)}
   \bi 
   \item Create the results matrix as in the previous slide
   \item Use \mcode{xlswrite} to create a spreadsheet of the results matrix
   \item Edit it further in Excel if needed
   \item Pro: Can do this in batch mode
   \item Con: Need to do a bit of editing on the Excel side (inconvenient when specification changes)
   \ei
\end{frame}

\begin{frame}
\frametitle{Matlab to LaTeX (method 1)}
   \bi 
   \item Download \mcode{matrix2latex} on the Matlab file exchange
   \item Save the results matrix as a LaTeX matrix
   \item Input it into your tex file
   \item Pro: This function does the LaTeX formatting for you
   \item Con: variable names are not included in the results matrix, so you need to find a way to incorporate these
   \ei
\end{frame}

\begin{frame}
\frametitle{Matlab to LaTeX (method 2)}
   \bi 
   \item Use \mcode{fprintf} to code the raw LaTeX version of your results table, variable names and all
   \item Pro: All LaTeX source code is right in front of you
   \item Con: More work if you decide to change the estimation specification (since you now need to change the LaTeX formatted table)
   \ei
\end{frame}

\begin{frame}
\frametitle{Other ways of formatting}
   \bi 
   \item Create a cell array that has variable names in it already
   \item Create another cell array with column headers
   \item Create another cell array with numerical estimation results
   \item Concatenate these cell arrays into a readable array of strings and numbers
   \item Export this to Excel or LaTeX in the ways previously discussed
   \item Pro: Cell array looks exactly how you want it to
   \item Cons: variable names have quotes around them; a few extra lines of code to form the cell array
   \ei
\end{frame}

\begin{frame}
\frametitle{Other ways of formatting}
   \bi 
   \item Use \mcode{mprint} (user-created Matlab function) to print the results
   \item Pro: results look like Stata output
   \item Con: can't get it into LaTeX very easily since it's formatted as plain text
   \ei
\end{frame}

\begin{frame}
\frametitle{Converting Excel to LaTeX}
   \bi 
   \item There is an Excel utility called excel2latex which will convert any subset of an Excel spreadsheet into a LaTeX table
   \item It works quite well and is something I have used extensively
   \ei
\end{frame}
\end{document}
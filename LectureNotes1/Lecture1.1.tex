\documentclass[english,xcolor=dvipsnames]{beamer}
% load package with ``framed'' and ``numbered'' option.
\usepackage[numbered,framed,autolinebreaks,useliterate]{mcode}
\usepackage[orientation=landscape,size=custom,width=16,height=9,scale=0.48,debug]{beamerposter}
\usepackage[T1]{fontenc}
\usepackage[latin9]{inputenc}
\usepackage{amsthm}
\usepackage{amsmath}
\usepackage{amssymb}
\usepackage{bookmark}
\usepackage{graphics,graphicx}
\usepackage{pstricks,pst-node,pst-tree}
\usefonttheme{serif}
\usepackage{palatino}
\usepackage{tikz}
\usetikzlibrary{shapes,arrows}
\usetikzlibrary{positioning}
%\usepackage[margin=.5cm]{geometry}

\definecolor{dgreen}{rgb}{0.,0.6,0.}
\definecolor{forest}{RGB}{34.,139.,34.}
\definecolor{byublue}{RGB}{0.,30.,76.}
\definecolor{dukeblue}{RGB}{0.,0.,156.}
%\usetheme{Ilmenau}
\usetheme{Warsaw}
\usecolortheme[named=dukeblue]{structure}
%\usecolortheme[named=RawSienna]{structure}
%\usecolortheme[named=byublue]{structure}
\setbeamertemplate{navigation symbols}{}
\setbeamertemplate{footline}{}
\setbeamercovered{transparent}

%%%%%%%%%%%%%%%%%%%%%%%%%%%%%%%%%%%%%%%%%%%%%%%%
% NOTE: With Ilmenau style, to get the bullets %
% looking right, do one section and one sub-   %
% section for each set of bullets              %
%%%%%%%%%%%%%%%%%%%%%%%%%%%%%%%%%%%%%%%%%%%%%%%%

\begin{document}
\begin{frame}
\title{Introduction to Matlab}
\author{
	Tyler Ransom\thanks{Slides adapted with permission from Justin Trogdon.}\\
	\emph{Duke University}\\
%    \today \\
%    \vspace{10cm}
}
\titlepage
\end{frame}

\begin{frame}{Outline}
\begin{itemize}
	\item Matlab user interface
	\item Accessing Matlab at Duke
	\item How Matlab works
	\item Operators
	\item Elementary Matrices
	\item Flow Control
	\item Loops
	\item Working with Data
	\item Help in Matlab
	\item Functions
\end{itemize}
\end{frame}

\begin{frame}{User Interface}
There are five main windows on the user interface:
	\begin{enumerate}
		\item \textbf{Command Window}: Interactive command line where user can create matrices, perform calculations, etc.
		\item \textbf{Workspace}: List of variables that are available to the user
		\item \textbf{Command History}: Lists recent commands issued by the user
		\item \textbf{Current Folder}: Gives a directory listing of the current directory (User can change the directory at the top of the screen, or by typing ``cd '' in the command window)
		\item \textbf{Variable Editor}: Allows user to view specific observations of a variable
	\end{enumerate}
\end{frame}

\begin{frame}{Accessing Matlab at Duke}
Four ways:
	\begin{enumerate}
		\item Windows machines in the Bowling Alley
		\begin{itemize}
			\item Log in and double-click on Matlab icon (make sure to change directory to H:\textbackslash)
		\end{itemize}
		\item Linux machines in the Bowling Alley
		\begin{itemize}
			\item Log in, open the terminal and type ``\texttt{matlab \&}'' or ``\texttt{matlab -nodisplay}'' (batch mode)
		\end{itemize}
		\item SSH terminal from a laptop or non-Duke desktop
		\begin{itemize}
			\item Log in, type ``\texttt{matlab}'' (interactive) or ``\texttt{matlab -nodisplay}'' (batch)
			\item Interactive mode typically runs very slowly when not on the Duke network
			\item Batch mode runs very quickly but only has the command prompt, so the user can't see the workspace or command history
		\end{itemize}
		\item Personal license
		\begin{itemize}
			\item Students can purchase a license of Matlab (Student Version) from MathWorks for \$100. The license is valid so long as it's being used for coursework. Non-student personal licenses are also available for purchase, but can be quite expensive.
		\end{itemize}
	\end{enumerate}
\end{frame}

\begin{frame}{How Matlab Works}
\begin{itemize}
	\item User gives it a text file (.m), which the program interprets line-by-line
	\item This is equivalent to evaluating each line at the Command Line
\end{itemize}
\vspace{1cm}
There are two types of m-files in Matlab:
\begin{enumerate}
	\item \textbf{Script files}: Series of command line commands (read line-by-line)
	\item \textbf{Function files}: Take arguments as inputs and give outputs
\end{enumerate}
\vspace{1cm}
Remember that nothing a user does is reproducible unless it is contained in a .m file.
\end{frame}

\begin{frame}[fragile]{Operators}
\begin{lstlisting}
%          %in-line comment
;          %suppresses output
A(:,1)     %references the first column and all rows of A
A(1,:)     %references the first row and all columns of A
A = []     %creates an empty matrix named A
A(index,:) %subsets the rows of A by some logical vector (index)
[A; B]     %vertically concatenates matrices A and B
[A B]      %horizontally concatenates matrices A and B
A.B        %creates a structure A with sub-matrix B
A'         %transpose of A
A+B        %matrix addition of A and B
A-B        %matrix subtraction of B from A
A*B        %matrix multiplication of A and B
\end{lstlisting}
Always use brackets to concatenate matrices and parentheses to reference matrix elements
\end{frame}

\begin{frame}[fragile]{Operators}
\begin{lstlisting}
A^B       %matrix exponentiation; B must be a scalar and A must be square
A.*B      %element-by-element multiplication of A and B (must be same size)
A\B       %postmultiplies A inverse by B
A/B       %postmultiplies A by B inverse
kron(A,B) %Kronecker product of A and B
A.^B      %element-by-element exponentiation of A by B
(A <= B)   %creates a logical array that is 1 if A<=B, 0 otherwise
(A == B)  %creates a logical array that is 1 if A=B, 0 otherwise
(A >= B)   %creates a logical array that is 1 if A>=B, 0 otherwise
A & B     %creates a logical array that is 1 if both A and B have non-zero elements at that location
A | B     %creates a logical array that is 1 if either A or B have non-zero elements at that location
~A        %same as A==0
\end{lstlisting}
Note that parentheses can also be used to create logical variables
\end{frame}

\begin{frame}[fragile]{Elementary Matrices}
\scriptsize{Matlab has commands to create ``common'' matrices so that users don't have to hand-code them}
%\vspace{-.2cm}
\begin{lstlisting}
eye(n)     %n-by-n identity matrix
ones(n,m)  %n-by-m matrix with the number 1 in every element
zeros(n,m) %n-by-m matrix with the number 0 in every element
m:step:n   %row vector from m to n increasing (or decreasing) by step each time
rand(n,m)  %n-by-m matrix of Uniform[0,1] random numbers
randn(n,m) %n-by-m matrix of Normal(0,1) random numbers
mvnrnd(mu,sigma) %matrix of multivariate Normal(mu,sigma) random numbers (dimension is same as mu)
length(A)  %lists number of elements of the longest dimension of A
size(A,d)  %gives the number of elements of the dth dimension of A
diag(A)    %captures diagonal elements of A if A is square. If A is a vector, creates a diagonal matrix from the elements of A
triu       %replaces lower triangular elements of A with zeros
tril       %replaces upper triangular elements of A with zeros
find(A>1)  %finds, e.g. elements of A that are larger than 1
\end{lstlisting}
\scriptsize{Note that all ``(n,m)'' commands can be generalized to more than two dimensions by using ``(n,m,p,...,z)''}
\end{frame}

\begin{frame}[fragile]{Matrix Relations}
Commands that help users know the properties of matrices in the workspace
\begin{lstlisting}
isreal(A)   %returns a 1 if all elements of A are real numbers
isempty(A)  %returns a 0 if A has at least one element
issparse(A) %returns a 1 if A is sparse (has many zeros)
isnan(A)    %returns a matrix of 0s or 1s corresponding to the elements of A that are not a number (NaN)
isfinite(A) %returns a matrix of 0s or 1s corresponding to the elements of A that are infinite
\end{lstlisting}
\end{frame}

\begin{frame}[fragile]{Special Variables}
Special variables reserved by Matlab
\begin{lstlisting}
ans     %the value (answer) of the most recent command line input
inf     %positive infinity
-inf    %negative infinity
NaN     %missing value (Not a Number) -- this is case sensitive
eps     %machine epsilon; equal to 2^-52
realmax %largest number Matlab can recognize; equal to 10^308
realmin %smallest number Matlab can recognize; equal to 10^-308
\end{lstlisting}
Note also that Matlab can do other mathematical functions on matrices (e.g. $\log(A)$ takes the $\log$ of all elements of A)
\end{frame}

\begin{frame}[fragile]{Flow Control}
\begin{itemize}
	\item Suppose a user wants to control which part of the script is executed
	\item This technique is referred to as \emph{flow control}
	\item Flow control in Matlab is very similar to other languages (C, Fortran)
\end{itemize}
\end{frame}

\begin{frame}[fragile]{Flow Control - if}
\begin{lstlisting}
if [some condition]
   [some code];
elseif [some other condition]
   [some code];
else
   [some code];
end
\end{lstlisting}
Conditions must be logical statements.\\
\mcode{elseif} and \mcode{else} statements are optional
\end{frame}

\begin{frame}[fragile]{Flow Control - switch}
\begin{lstlisting}
switch [some expression]
case [some case]
   [some code];
case [some other case]
   [some code];
otherwise
   [some code];
end
\end{lstlisting}
Case expressions are either scalar numbers or strings (not logical statements).\\
Matlab goes through each case expression and executes the code below it if it matches the switch expression.\\
I have never used \mcode{switch}, but it is useful if the user needs to execute different code depending on a finite number of differing case scenarios
\end{frame}

\begin{frame}[fragile]{Flow Control - error and return}
A user can check that code is running according to plan by forcing Matlab to give an error if something is amiss
\begin{lstlisting}
if [condition that should hold doesn't]
   error('This code doesn`t work');
end
\end{lstlisting}
A user can issue the \mcode{return} command to force Matlab to exit the script
\end{frame}

\begin{frame}[fragile]{Looping}
\begin{itemize}
	\item Matlab has two types of loops: \mcode{for} and \mcode{while} 
	\item These loops literally execute commands repeatedly as long as the looping conditions are satisfied
	\item Loops in Matlab are useful for mechanizing code, but very slow because they are interpreted line by line
	\item \mcode{for} and \mcode{while} are basically the same, but have subtle differences
	\item Loop syntax:
\end{itemize}
\begin{lstlisting}
for loopvar = starting_value:stepsize:stopping_value
    [code user would like to repeat]
end

while [looping condition]
      [code user would like to repeat]
end
\end{lstlisting}
\end{frame}

\begin{frame}[fragile]{Looping: \texttt{for} vs. \texttt{while}}
Use a \mcode{for} loop when
\begin{itemize}
	\item You know a command should be execute a certain number of times
	\item You can't use a built-in Matlab command to accomplish your design
\end{itemize}
\vfill
Use a \mcode{while} loop when
\begin{itemize}
	\item You don't know how many times the commands need to be repeated
	\item You can't use a built-in Matlab command to accomplish your design
\end{itemize}
\end{frame}

\begin{frame}[fragile]{Looping Example}
\begin{itemize}
	\item Suppose a user wants to ``mechanize'' how she is creating a matrix. For example, suppose she wants to create a matrix $B$ that is $2^k$ for each row $k$ in the first column, and $\ln(k)$ for each row $k$ in the second column. 
	\item If $k$ is small, she could hand code it:
\end{itemize}
\begin{lstlisting}
B(1,1) = 2^1;
B(1,2) = log(1);
B(2,1) = 2^2;
B(2,2) = log(2);
\end{lstlisting}
\end{frame}

\begin{frame}[fragile]{Looping Example - Continued}
\begin{itemize}
	\item However, if $k$ is large, hand coding is tedious and more vulnerable to user mistakes.
	\item Instead, the user can write a \mcode{for} loop to accomplish the task:
\end{itemize}
\begin{lstlisting}
B = zeros(1e5,2); %initialize the size of the matrix for speed
for k=1:length(B) %allow the loop length to change automatically according to the size of B
    B(k,1) = 2^k;
    B(k,2) = log(k);
end
\end{lstlisting}
\begin{itemize}
	\item As you can see, loops are very powerful in performing repetitive tasks
	\item Matlab, however, is slow at doing loops (since it interprets commands line-by-line), so loops should generally be avoided if at all possible
\end{itemize}
\end{frame}

\begin{frame}[fragile]{When not to loop}
\begin{itemize}
	\item Suppose a user wants to delete rows in a vector $V$ if elements of $V$ are larger than 5
	\item Two ways to do this:
\end{itemize}
\begin{lstlisting}
for i=1:length(V)
    if V(i)>5;
       V(i)=[];
    end
end
\end{lstlisting}
\begin{itemize}
	\item A much easier way to do this is to make use of Matlab's logical indexing capabilities
\end{itemize}
\begin{lstlisting}
V(V>5)=[];
\end{lstlisting}
\end{frame}


\begin{frame}[fragile]{While loops}
Consider the \mcode{for} loop from before
\begin{lstlisting}
B = zeros(1e5,2); %initialize the size of the matrix for speed
for k=1:length(B) %allow the loop length to change automatically according to the size of B
    B(k,1) = 2^k;
    B(k,2) = log(k);
end
\end{lstlisting}
We can accomplish the same task using a \mcode{while} loop:
\begin{lstlisting}
B = zeros(1e5,2);
k = 1
while k<=length(B) %Note the use of the weak inequality
    B(k,1) = 2^k;
    B(k,2) = log(k);
    k = k+1;
end
\end{lstlisting}
\end{frame}

\begin{frame}[fragile]{While loops - continued}
\begin{itemize}
	\item In their simplest form, \mcode{while} loops and \mcode{for} loops look very similar and accomplish the same task
	\item To illustrate the (subtle) difference, suppose a user wants to repeatedly execute code until some criterion is met
	\item However, the user may not know how many times the code should be repeated
	\item The \mcode{while} loop takes care of this in a way the \mcode{for} loop can't
\end{itemize}
\begin{lstlisting}
% This code takes a matrix A and element-wise multiplies it by a matrix F, then divides by the loop counter until its norm reaches zero
A = rand(15);
F = eye(size(A));
i = 1;
while norm(F,1) > 0,
   F = A.*F/i;
   i = i + 1;
end
\end{lstlisting}
\end{frame}

\begin{frame}[fragile]{Looping Summary}
\begin{itemize}
	\item Subsetting \mcode{if} statements in Matlab should always be taken care of through logical indexing
	\item There are some cases when looping can't be avoided (e.g. merging two datasets, creating matrices that systematically vary by element location [like the example from two slides previous])
	\item \mcode{while} loops are useful for executing commands where the number of iterations is not known from the outset (e.g. contraction mappings)
	\item In fact, structural economists use \mcode{while} loops often because many problems can be characterized by a two-step estimation procedure which the \mcode{while} loop is especially tailored to deal with
\end{itemize}
\end{frame}

\begin{frame}{Working with data in Matlab}
\begin{itemize}
	\item Matlab-formatted data comes in a .mat file
	\item \mcode{load mydata.mat} loads the data onto the workspace
	\item \mcode{load mydata.mat var1 var2 var3} loads variables \mcode{var1 var2 var3} of \mcode{mydata.mat} onto the workspace
	\item Be careful about which directory you're in when loading data. Matlab usually opens to its default directory, which is generally not the directory a user is working in
	\item \mcode{save} works just like load, except it saves variables in the workspace into a .mat file
	\item e.g. \mcode{save mydata1.mat var1 var2 var3}
\end{itemize}
\end{frame}

\begin{frame}{Loading non-Matlab-formatted data}
\begin{itemize}
	\item The easiest way to import non-mat files is through the import wizard (Start$\blacktriangleright$Matlab$\blacktriangleright$Import Wizard) in the lower left corner of the Matlab interface
	\item If importing from other software (e.g. Stata or SAS), easiest way is to go from .dta$\blacktriangleright$.csv$\blacktriangleright$.mat
	\item StatTransfer also supports Matlab, though it is unavailable on Econ department machines (but available at Perkins Library)
	\item Matlab does not allow string components in data matrices, so make sure you don't export value labels from other software
	\item Also, users may need to manually rename variables once the data matrices have been imported
\end{itemize}
\end{frame}

\begin{frame}{Reproducibly importing data}
\begin{itemize}
	\item The import wizard gives users the option to generate m-code that is reproducible; however, I've never been successful in re-running this reproducible code
	\item Matlab has built-in functions to import .csv or fixed-format file types:
	\item \mcode{csvread('xyz.csv')}
	\item \mcode{csvwrite('xyz.csv', A)}
  \item \mcode{dlmread('xyz.dat', 'delimiter string')}
  \item \mcode{dlmwrite('xyz.dat', A, 'delimiter string')}
  \item \mcode{A=importdata('xyz.csv', delimiter, nheaderlines)} is the best way to import data, because it automatically converts unreadable elements to \mcode{NaN}
\end{itemize}
\end{frame}

\begin{frame}{Dealing with missing values}
\begin{itemize}
	\item Most of the problems in data conversion lie in mapping missing values from one software to another
	\item Missing values are coded as \mcode{NaN} in Matlab; in Stata or SAS they are a dot (\mcode{.})
	\item StatTransfer automatically converts \mcode{.} to \mcode{NaN}
	\item Without StatTransfer, need to do a find-and-replace of \mcode{.} (or blank) to \mcode{NaN} prior to importing to Matlab (unless using \mcode{importdata})
	\item \mcode{importdata} does this automatically
	\item Matlab 2011 has an import utility that allows the user to convert non-numeric values to NaN within the Matlab import wizard
\end{itemize}
\end{frame}

\begin{frame}[fragile]{Describing data}
\begin{lstlisting}
[val,idx]=sort(A)     %sorts each column of A in ascending order
[val,idx]=sortrows(A) %same as sort, but preserves row groupings (just like sorting in Stata or SAS)
[val,idx]=unique(A)   %strips out unique elements of A (no repeats)
[val,idx]=max(A)      %max element of A
[val,idx]=min(A)      %min element of A
mean(A)               %mean
median(A)             %median
var(A)                %variance
std(A)                %standard deviation
sum(A)                %summation
hist(A)               %histogram
corrcoef(A)           %correlation coefficient of A
cov(A)                %covariance of A
\end{lstlisting}
Note that summary statistics can be extended to multiple dimensions\\
e.g. \mcode{max(A,2)} is the maximum across columns of A (instead of down rows)
\end{frame}

\begin{frame}[fragile]{Describing data with missing observations}
Summary statistics from the previous slide will return \mcode{NaN} if any element of the vector (or matrix) is \mcode{NaN}. There are two ways to get around this:
\begin{itemize}
	\item \mcode{mean(~isnan(A))} returns the mean ignoring missing values
	\item A faster way is to use \mcode{nanmean(A)}, which accomplishes the same task
	\item Most other commands listed in the previous slide have ``nan'' counterparts
	\item Other built-in functions will explicitly mention in the help menu that they ignore \mcode{NaN} values
\end{itemize}
\end{frame}

\begin{frame}[fragile]{Accessing Matlab's help}
There are a number of ways to view documentation on a certain command, or find commands that accomplish something of interest:
\begin{itemize}
	\item Type \mcode{help commandname} in the command window if you know the name of a command , but want to learn more about its syntax or usage
	\item Type \mcode{lookfor word} in the command window to get a list of Matlab commands that have \mcode{word} in the command description
	\item Google ``matlab commandname'' for help on the \mcode{commandname} command (Matlab's official documentation for the command is usually the top hit)
	\item Google something more general like ``how to sort rows in matlab.'' This will usually bring up the Matlab documentation, but may also lead to valuable discussion sites like StackOverflow
\end{itemize}
It is rare to come across a problem or question that no one else has already thought of or solved
\end{frame}

\begin{frame}[fragile]{Functions}
\begin{itemize}
	\item Functions are an integral part of Matlab and must be used when estimating some econometric models (e.g. MLE)
	\item Functions are powerful in their own right because they allow users to replicate code in a systematic fashion
	\item Function syntax:
\end{itemize}
\begin{lstlisting}
[output1,output2,..,outputN] = function_name(input1,input2,..,inputK)
% Comments that describe what the function does
[code which user would like to have repeatedly executed and which uses all inputs and creates all outputs]
end
\end{lstlisting}
\end{frame}

\begin{frame}{Using functions}
% Define block styles
\tikzstyle{block} = [rectangle, draw, fill=blue!20, 
    text width=16em, rounded corners, minimum height=4em]
\tikzstyle{line} = [draw, -latex']
\tikzstyle{block2} = [rectangle, draw, fill=red!20, 
    text width=16em, rounded corners, minimum height=4em]
The following diagram illustrates how functions are called in Matlab:
\begin{tikzpicture}[node distance = .5cm, auto]
\tikzstyle{every node}=[font=\small]
    % Place nodes
    \node [block] (script) {\textbf{Script m-file (\texttt{example.m})} \\ . \\ . \\ . \\ \texttt{B=my\_function(A)}};
    \node [block2, below right = of script] (function) {\textbf{Function m-file (\texttt{my\_function.m})} \\ \texttt{out1=my\_function(inp1)}\\  \texttt{out1 = (inp1>0);} \\ \texttt{end}};
    \node [block, below left = of function] (script2) {\textbf{Script m-file (continued)} \\ \texttt{B=my\_function(A)} \\ . \\ . \\ . };

    % Draw edges
    \path [line] (script) -- (function);
    \path [line] (function) -- (script2);
\end{tikzpicture}
\end{frame}

\begin{frame}{Using functions - Continued}
Helpful reminders when working with functions
\begin{itemize}
	\item Always name a function m-file the same name as the function title
	\item Once Matlab enters a function, only function inputs are available in the workspace
	\begin{itemize}
		\item The exception to this is that all \mcode{global} variables are passed to the function
	\end{itemize}
	\item Once Matlab re-enters the script m-file (after executing the function), only the function output is available (in addition to the original workspace that existed before the function call)
	\item Other functions can be referenced within functions. Users may also create an arbitrary number of nested functions (i.e. functions within functions)---I typically don't do this
\end{itemize}
\end{frame}

\begin{frame}[fragile]{Helpful function variables}
\begin{itemize}
	\item These are only valid inside a function m-file
	\item Most useful for implementing flow control in a function
\end{itemize}
\begin{lstlisting}
nargin    %lists the number of inputs to the function
nargout   %lists the number of outputs from the function
varargin  %shorthand for ''all inputs after this point''
varargout %shorthand for ''all outputs after this point''
\end{lstlisting}
\end{frame}

\begin{frame}[fragile]{Accessing function outputs}
\begin{itemize}
	\item Suppose a user only wants to see the first and fifth outputs of a function
	\item Matlab 2009 and later allows the use of $\sim$ to suppress function outputs, e.g. \mcode{[x,~,~,~,z] = myfun(X,Y,Z);}
\end{itemize}
\end{frame}

\begin{frame}{Function handles}
\begin{itemize}
	\item Matlab allows users to create functions ``on the fly'' (without needing to create a m-file)
	\item A function handle is a function without a name, which usually has a simple expression (and only a few inputs)
	\item The function handle syntax is \mcode{@(x) f(x)}
	\item e.g. To create a function handle for $f(x)=x^2 + 2x - 8$, the correct syntax would be \mcode{@(x) x.^2+2.*x-8}
	\item Note that function handle operators need to be vectorized (\mcode{.*}, \mcode{.^})
	\item Can also be functions of multiple variables
\end{itemize}
\end{frame}

\begin{frame}{Variable scope}
\begin{itemize}
	\item As was mentioned previously, when Matlab enters a function it leaves behind the workspace from the file (or function) it came from
	\item Users can make certain variables available to all functions by issuing a \mcode{global} statement at the top of each file
	\item All variables that are not contained in the \mcode{global} statement are not transferred
	\item Changes to a \mcode{global} variable in one function are brought into the the next function, so be careful!
	\item This is useful in situations where a user needs to pass a large number of variables and doesn't want to create a function with a large number of inputs
\end{itemize}
\end{frame}

\begin{frame}{Example 1: Three ways to evaluate code}
Suppose that a user wants to calculate the product $Z$ of two matrices $X$ and $Y$. She could do this in three ways:
\begin{enumerate}
	\item Type into the command line
	\item Save as a line in a script m-file
	\item Create a function to do the product
\end{enumerate}
\end{frame}

\begin{frame}{Example 1 -- Command Line}
Supposing $X$ and $Y$ already exist, type \mcode{Z=X*Y;} into the command line, and press enter.
\end{frame}

\begin{frame}{Example 1 -- Script}
Supposing $X$ and $Y$ already exist, type \mcode{Z=X*Y;} into line 1 of the editor.
\end{frame}

\begin{frame}[fragile]{Example 1 -- Function}
Create a new script and type the following:
\begin{lstlisting}
Z = matrixproduct(X,Y)
% This function calculates the product of two matrices (X and Y)
Z=X*Y;
end
\end{lstlisting}
Save the script as \mcode{matrixproduct.m}
\vspace{1cm}
In a new script, type \mcode{Z=matrixproduct(X,Y);}
\end{frame}


\end{document}
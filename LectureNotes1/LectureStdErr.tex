\documentclass[english,xcolor=dvipsnames]{beamer}
% load package with ``framed'' and ``numbered'' option.
\usepackage[autolinebreaks,useliterate]{mcode}
\usepackage[orientation=landscape,size=custom,width=16,height=9,scale=0.48,debug]{beamerposter}
\usepackage[T1]{fontenc}
\usepackage[latin9]{inputenc}
\usepackage{amsthm}
\usepackage{amsmath}
\usepackage{amssymb}
\usepackage{bookmark}
\usepackage{verbatim}
\usepackage{graphics,graphicx}
\usepackage{pstricks,pst-node,pst-tree}
\usefonttheme{serif}
\usepackage{palatino}
%\usepackage[margin=.5cm]{geometry}

\definecolor{dgreen}{rgb}{0.,0.6,0.}
\definecolor{forest}{RGB}{34.,139.,34.}
\definecolor{byublue}{RGB}{0.,30.,76.}
\definecolor{dukeblue}{RGB}{0.,0.,156.}
%\usetheme{Ilmenau}
\usetheme{Warsaw}
\usecolortheme[named=dukeblue]{structure}
%\usecolortheme[named=RawSienna]{structure}
%\usecolortheme[named=byublue]{structure}
\setbeamertemplate{navigation symbols}{}
\setbeamertemplate{footline}{}
\setbeamercovered{transparent}

%%%%%%%%%%%%%%%%%%%%%%%%%%%%%%%%%%%%%%%%%%%%%%%%
% NOTE: With Ilmenau style, to get the bullets %
% looking right, do one section and one sub-   %
% section for each set of bullets              %
%%%%%%%%%%%%%%%%%%%%%%%%%%%%%%%%%%%%%%%%%%%%%%%%

\begin{document}
\begin{frame}
\title{Standard Errors in Matlab}
\author{
	Tyler Ransom\\
	\emph{Duke University}\\
%    \today \\
%    \vspace{10cm}
}
\titlepage
\end{frame}

\begin{frame}{Outline}
\begin{itemize}
	\item Importance of Standard Errors
	\item Recovering SEs in GMM models
	\item Recovering SEs in MLE models
	\item Recovering SEs in other models 
\end{itemize}
\end{frame}

\begin{frame}{Why all the fuss over standard errors?}
\begin{itemize}
	\item Empiricists are in the business of estimating models and assigning meaning to those model estimates
	\item Coupled with point estimates, standard errors are the most crucial piece of an empiricist�s findings
	\item If the standard errors are wrong, then the results will be wrong, too, because false inferences will be made
	\item This is why so much energy is spent deriving correct standard errors under a variety of assumptions (e.g. Hayashi�s textbook)
\end{itemize}
\end{frame}

\begin{frame}{Standard errors}
Economists prefer to have models with parameter estimates that are \emph{consistent} and \emph{asymptotically normal}, i.e.\\
\begin{equation*}
\sqrt{n}\left(\hat{\beta}-\beta\right) \overset{d}{\rightarrow} N\left(0,\textrm{Avar}\right)
\end{equation*}
(Hayashi, p. 113)
\begin{itemize}
	\item We want to know the variance of our parameter estimates so that we can tell whether or not we got a lucky draw, or if they are actually different from some number (typically zero) 
	\item The standard errors of the $\hat{\beta}$ vector are embedded in the $\widehat{\textrm{Avar}}$ matrix (our estimate of the true $\textrm{Avar}$ matrix)
	\item In the next few slides, we'll go over what the formula is for $\widehat{\textrm{Avar}}$ in various common econometric models
\end{itemize}
\end{frame}

\begin{frame}{Standard errors for GMM models (Hayashi pp. 125, 213)}
$\widehat{\textrm{Avar}}=\left(\mathbf{Z^{\prime}X}\widehat{\mathbf{S}}^{-1}\mathbf{X^{\prime}Z}\right)^{-1}$\\
where the optimal weighting matrix ($\textrm{plim}\,\mathbf{\widehat{W}=S}^{-1}$) is used, and\\
\begin{eqnarray*}
\mathbf{\widehat{S}} & = & \sum_{i=1}^{n}\hat{\varepsilon}_{i}^{2}\mathbf{x_{i}x_{i}^{\prime}}\\
 & = & \mathbf{X^{\prime}BX}
\end{eqnarray*}
where
\[
\mathbf{B}=\left[\begin{array}{ccc}
\hat{\varepsilon}_{1}^{2}\\
 & \ddots\\
  &  & \hat{\varepsilon}_{n}^{2}
\end{array}\right]
\]
so
\[
\widehat{\textrm{Avar}}=\left(\mathbf{Z^{\prime}X}\left(\mathbf{X^{\prime}BX}\right)^{-1}\mathbf{X^{\prime}Z}\right)^{-1}
\]
\end{frame}

\begin{frame}{Standard errors for GMM models}
From the previous slide:
\[
\widehat{\textrm{Avar}}=\left(\mathbf{Z^{\prime}X}\left(\mathbf{X^{\prime}BX}\right)^{-1}\mathbf{X^{\prime}Z}\right)^{-1}
\]
where
\begin{itemize}
	\item $\mathbf{X}$ is a matrix of instruments
	\item $\mathbf{Z}$ is a matrix of endogenous regressors
	\item $\mathbf{B}_{i,i} = \left(y_{i}-x_{i}\hat{\beta}\right)^{2}$ 
	\item Note: for simple OLS, substitute $\mathbf{X}$ for $\mathbf{Z}$ in the above formula
\end{itemize}
The standard errors are then
\[
\textrm{SE}_{\hat{\beta}_{k}} = \sqrt{\widehat{\textrm{Avar}}_{k,k}}
\]
\end{frame}

\begin{frame}{Standard errors for MLE models (Hayashi pp. 475-476)}
\[
\widehat{\textrm{Avar}}=-\left\{ \sum_{i=1}^{n}\mathbf{H}\left(\mathbf{x_{i}};\mathbf{\widehat{\beta}}\right)\right\} ^{-1}
\]
where
\[
\mathbf{H}\left(\mathbf{x_{i}};\mathbf{\widehat{\beta}}\right)=\frac{\partial^{2}\ell\left(x_{i},\beta\right)}{\partial \beta \partial \beta ^{\prime}}
\]
where $\ell\left(x_{i},\beta\right)$ is the log likelihood function
\end{frame}


\begin{frame}[fragile]{Standard errors for MLE models}
Recall in Matlab that \mcode{fminunc} and \mcode{fmincon} can return the numerical hessian of the objective function:
\begin{lstlisting}
[x,fval,exitflag,output,gradient,hessian] = fminunc('fun',x0,options,varargin)
\end{lstlisting}
\\
Assuming that all regularity conditions are met, the standard errors of \mcode{x} are
\begin{lstlisting}
se = sqrt(diag(inv(hessian)));
\end{lstlisting}
since Matlab returns $-\left\{ \sum_{i=1}^{n}\mathbf{H}\left(\mathbf{x_{i}};\mathbf{\widehat{\beta}}\right)\right\}$ in the \mcode{hessian} output
\end{frame}

\begin{frame}[fragile]{Standard errors for other models}
\begin{itemize}
	\item All classical econometric models can be considered as either GMM or MLE. (Furthermore, GMM and MLE are asymptotically equivalent in some situations)
	\item Most structural econometric models, however, incorporate simulation methods, multiple estimation stages, and/or contraction mapping/fixed point algorithms (in addition to a main GMM/MLE optimization)
	\item In these more complex estimation algorithms, it's unclear how the standard errors of the parameter estimates are affected by the use of previous stage estimates and/or simulation draws
	\item The solution to this problem is bootstrapping, which we will cover in the second Matlab module
\end{itemize}
\end{frame}

\end{document}
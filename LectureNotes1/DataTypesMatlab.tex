\documentclass[english,xcolor=dvipsnames]{beamer}
% load package with ``framed'' and ``numbered'' option.
\usepackage[numbered,framed,autolinebreaks,useliterate]{mcode}
\usepackage[orientation=landscape,size=custom,width=16,height=9,scale=0.48,debug]{beamerposter}
\usepackage[T1]{fontenc}
\usepackage[latin9]{inputenc}
\usepackage{amsthm}
\usepackage{amsmath}
\usepackage{amssymb}
\usepackage{bookmark}
\usepackage{graphics,graphicx}
\usepackage{pstricks,pst-node,pst-tree}
\usefonttheme{serif}
\usepackage{palatino}
\usepackage{tikz}
\usetikzlibrary{shapes,arrows}
\usetikzlibrary{positioning}
%\usepackage[margin=.5cm]{geometry}

\definecolor{dgreen}{rgb}{0.,0.6,0.}
\definecolor{forest}{RGB}{34.,139.,34.}
\definecolor{byublue}{RGB}{0.,30.,76.}
\definecolor{dukeblue}{RGB}{0.,0.,156.}
%\usetheme{Ilmenau}
\usetheme{Warsaw}
\usecolortheme[named=dukeblue]{structure}
%\usecolortheme[named=RawSienna]{structure}
%\usecolortheme[named=byublue]{structure}
\setbeamertemplate{navigation symbols}{}
\setbeamertemplate{footline}{}
\setbeamercovered{transparent}

\begin{document}
\begin{frame}
\title{Data Types in Matlab}
\author{
	Tyler Ransom\\
	\emph{Duke University}\\
%    \today \\
%    \vspace{10cm}
}
\titlepage
\end{frame}

\begin{frame}{Data Types in Matlab}
\begin{itemize}
\item Numeric Classes
\item The Logical Class
\item Characters and Strings
\item Structures
\item Cell Arrays
\item Function Handles
\item Map Containers
\end{itemize}
\end{frame}


\begin{frame}{Why worry about data types?}
\begin{itemize}
\item Matlab makes it difficult to combine different data types
\item For numeric classes, data types may affect accuracy
\begin{itemize}
\item e.g. rounding error
\end{itemize}
\item It's important to know about data types when working with data in
any program, because different data types take up different amounts
of disk space
\item Don't waste disk space by saving a dummy variable in a format that
has precision to 15 decimal points
\end{itemize}
\end{frame}


\begin{frame}{Numeric Classes}
Integers
\begin{itemize}
\item can take 8, 16, 32, or 64 bits
\item can be signed or unsigned
\item if unsigned, user can get twice as long of integers per byte
\end{itemize}
Double Floating-Point Numbers
\begin{itemize}
\item always take up 64 bits
\item most precise numerical format (16 digits of accuracy)
\end{itemize}
Single Floating-Point Numbers
\begin{itemize}
\item always take up 32 bits
\item only 8 digits of accuracy
\end{itemize}
Complex Numbers
\begin{itemize}
\item can be formatted in any fashion above, but Matlab has an additional
flag to denote complexity
\end{itemize}
\texttt{Inf} and \texttt{NaN}
\begin{itemize}
\item stored as 64-bit double
\end{itemize}
\end{frame}


\begin{frame}[fragile]{Working with Numeric Classes}
\begin{itemize}
\item Viewing which variables are which type
\begin{itemize}
\item At the command line, type \texttt{whos}
\item User can see a list of variables and which class they are formatted
\end{itemize}
\item Generating a variable of a certain type:
\end{itemize}
\begin{lstlisting}
y = int64(-589324077574); % Create a 64-bit integer
x = double(y)             % Convert to double
\end{lstlisting}
\end{frame}


\begin{frame}{How can I optimize my data storage?}
\begin{itemize}
\item Suppose a user has data she wants to save in a mat-file
\item Suppose also that she has a number of variables that are stored in
{}``too big'' of numeric classes
\begin{itemize}
\item e.g. she has a dummy variable stored as a 32-bit integer and a 16-bit
integer stored as a 64-bit double
\end{itemize}
\item How can she quickly and easily convert her numerical variables to
the most efficient numerical class?
\item The easiest way is the \texttt{save} command
\item Matlab's \texttt{save} command is streamlined to automatically convert
variables to the optimal format
\item Beyond this, there are user-written scripts that store data even more
efficiently than Matlab's \texttt{save} command
\end{itemize}
\end{frame}


\begin{frame}{Logical Class}
\begin{itemize}
\item Any variable that is a logical (e.g. {}``\texttt{Z=(x<5);}'') is
stored as an 8-bit \emph{logical}.
\end{itemize}
\end{frame}


\begin{frame}{Characters and Strings}
\begin{itemize}
\item Strings in Matlab are stored as 16-bits per character
\item Character arrays must have all elements be the same length, so spaces
are automatically inserted to make the dimensions match
\item Thus, if you tried %\mcode{data = ['Allison Jones';'Development';'Phoenix'];}\\
you would get an error because the dimensions don't match up
\item Instead, use the \texttt{char} function:\\
%\mcode{data = char('Allison Jones','Development','Phoenix')}
\item Can convert strings to cell arrays (see below)
\end{itemize}
\end{frame}


\begin{frame}{Structures}
Structures are invoked with a period, e.g. (\texttt{A.data}). \texttt{A}
is a structure with substructure \texttt{data}.
\end{frame}


\begin{frame}{Cell Arrays}
Cell arrays are the only way to store data of two different types
in the same array, side-by-side.
\begin{itemize}
\item Cell arrays contain multiple data containers called cells
\item Each cell can contain any type of data
\item When referencing elements, use \texttt{()} to reference cells (data
containers) and use \texttt{\{\}} to reference individual elements
within a cell
\end{itemize}
\end{frame}


\begin{frame}{Operations with Cell Arrays}
\begin{itemize}
\item Generate a blank cell array \texttt{A} with the command \texttt{A=cell(3,4,2);}
\item concatenate cell arrays along dimension $k$ with the command \texttt{cat(k,array1,array2,...arrayN);}
\item delete the $(i,j)$ observation of cell array \texttt{myarray} with
the command \texttt{myarray\{i,j\}={[}{]};}
\item create a nested cell array called super\_array with the command \texttt{super\_array
= \{array1; array2;\};}
\begin{itemize}
\item super\_array is now a 2x1 cell array with array1 in the first cell
and array2 in the second cell
\end{itemize}
\end{itemize}
\end{frame}


\begin{frame}[fragile]{Operations with Cell Arrays}
\begin{itemize}
\item In order to turn a string into a cell array, use the command \texttt{cellstr}
\item In order to turn a numerical array into a cell array, use \texttt{num2cell}
\item Once this formatting is taken care of, the string and numerical cells
can be concatenated together with the \texttt{cat} command
\item Example: want to list variable names next to their point estimates
in a regression output table:
\end{itemize}
\begin{lstlisting}
varnames = cellstr(char('Var1','Var2',...,'VarK'));
estimate = num2cell(X\y);
result   = cat(2,varnames,estimate);
\end{lstlisting}
\end{frame}


\begin{frame}{Function Handles}
From the Mathworks website:
{}``A function handle is a callable association to a MATLAB function.
It contains an association to that function that enables you to invoke
the function regardless of where you call it from. This means that,
even if you are outside the normal scope of a function, you can still
call it if you use its handle.
With function handles, you can:
\begin{itemize}
\item Pass a function to another function
\item Capture data values for later use by a function
\item Call functions outside of their normal scope
\item Save the handle in a MAT-file to be used in a later MATLAB session''
\end{itemize}
\end{frame}


\begin{frame}{Map Containers}
\begin{itemize}
\item Map containers are data structures that are organized by a key and
a data value
\item In regular arrays, the key is a number (corresponding to the element
of the matrix)
\item The advantage to map containers is that the key can be a string
\end{itemize}
\end{frame}
\end{document}

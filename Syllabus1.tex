%% LyX 2.2.3 created this file.  For more info, see http://www.lyx.org/.
%% Do not edit unless you really know what you are doing.
\documentclass[12pt,english]{article}
\usepackage{mathptmx}
\usepackage[T1]{fontenc}
\usepackage[latin9]{inputenc}
\usepackage{geometry}
\geometry{verbose,tmargin=1in,bmargin=1in,lmargin=1in,rmargin=1in}
\usepackage{babel}
\usepackage[authoryear]{natbib}
\usepackage[unicode=true,pdfusetitle,
 bookmarks=true,bookmarksnumbered=false,bookmarksopen=false,
 breaklinks=false,pdfborder={0 0 0},pdfborderstyle={},backref=false,colorlinks=false]
 {hyperref}
\usepackage{breakurl}

\makeatletter

%%%%%%%%%%%%%%%%%%%%%%%%%%%%%% LyX specific LaTeX commands.
%% Because html converters don't know tabularnewline
\providecommand{\tabularnewline}{\\}

%%%%%%%%%%%%%%%%%%%%%%%%%%%%%% User specified LaTeX commands.
\date{}

\makeatother

\begin{document}

\title{\textbf{Foundations of Matlab: Summer II.1 2012}}

\maketitle
\vspace{-0.85in}

\begin{center}
\begin{tabular}{ll}
Course: & Foundations of Matlab (half Summer module for rising second-year PhD
students)\tabularnewline
Instructor: & Tyler Ransom\tabularnewline
Time: & Tue/Thu 10 a.m. - 12:30 p.m.; Friday 11 a.m. - 12 p.m.\tabularnewline
Location: & Social Sciences 113\tabularnewline
Office: & 2106 Campus Dr \#201A\tabularnewline
Email: & \href{mailto:tyler.ransom@duke.edu}{tyler.ransom@duke.edu}\tabularnewline
Office Hours: & By appointment\tabularnewline
\end{tabular}
\par\end{center}

\subsubsection*{About this course}

This course will build students' fundamental skills in Matlab in preparation
for 2nd-year PhD modules. Another name for it could be ``What I wish
I had learned before starting my second year of the PhD program.''
Students will become familiar with the Matlab's features and language
structure; learn general programming skills, data management techniques,
and how to describe data; and code a variety of extremum and maximum
likelihood estimators introduced in the first year of the PhD program
using Matlab's different functional optimizers. 

\subsubsection*{Prerequisites}

I am going to assume that all students are comfortable with OLS, logit
and probit models. These models will be discussed in lectures as if
students already have a working knowledge of them.

\subsubsection*{Textbooks}

There is no formal textbook for the course, but students may find
the following resources helpful: \emph{Econometrics} (Hayashi), \emph{Time
Series Analysis} (Hamilton), \emph{Econometric Analysis} (Greene),
\emph{Numerical Methods in Economics} (Judd), and \emph{Econometric
Analysis of Cross Section and Panel Data} (Wooldridge).

\subsubsection*{Registration, Enrollment and Overall Class Grades}

In order to get credit for this course, you need to first enroll in
Econ 360 or Econ 370 in ACES. Because each module is not listed separately
in ACES, I can only determine your enrollment in this course through
your completion of problem sets. In order to get credit for Econ 360
or Econ 370 in ACES, each student must enroll in at least two modules.
Grades from those two modules will be averaged into a final grade
for Econ 360/370. If you enroll in more than two modules, your final
grade will be an average of your two highest module grades. Because
of this favorable grading policy, you are invited to take more than
two modules.

\subsubsection*{Grades for this Module}

Grades will be determined by the average score from three problem
sets, due each week by 11:59 p.m. on Thursday. Late problem sets will
not be accepted. Submit problem set materials to your ``dropbox''
folder on Sakai. You are allowed to work on problem sets in groups
(no larger than 3, please), but each student must turn in his/her
own copy of the problem set. In particular, each student should avoid
copying/pasting code and instead type the code out on his/her own.
(This is the only way to learn how to program.) Put your name and
the names of those in your group at the top of your code file(s) and/or
writeup. Each Friday morning I will post solutions at 8 A.M. We will
then spend the Friday lecture time going through the code together.
Problem sets will be graded on the following scale (some convex combination
of effort and accuracy):

\medskip{}
\begin{center}
\begin{tabular}{cl}
4: & Problem set is complete and mostly correct\tabularnewline
3: & Problem set is complete with errors; or mostly complete and mostly
correct\tabularnewline
2: & Problem set is complete with many errors; or barely complete and mostly
correct\tabularnewline
1: & Problem set is barely attempted or completely incorrect\tabularnewline
0: & Problem set turned in late or not at all\tabularnewline
\end{tabular}\\
\par\end{center}

Problem set grades will be combined to an unweighted average to determine
course grade.

\subsubsection*{Schedule of Topics (subject to change)}
\begin{center}
\begin{tabular}{ccll}
\hline 
Class & Date & ~~~~~~~~~~~~~~~~~~~~~~~~~~~~~~Topics & Lecture Notes\tabularnewline
\hline 
1 & Tue 7/3 & Intro to Matlab interface; program structure; & Lec1.1.pdf\tabularnewline
 &  & loops; matrix operations; & \tabularnewline
 &  & introduction to functions/syntax & \tabularnewline
2 & Thu 7/5 & Functional optimizers; debugging; & MLEIntroNotes.pdf\tabularnewline
 &  & Data management; fminunc; fminsearch & Lec1.2.pdf\tabularnewline
 &  & \emph{PS1 due by 11:59 p.m.} & \tabularnewline
3 & Fri 7/6 & Go over code for Problem Set 1 & \tabularnewline
4 & Tue 7/10 & \LaTeX & OptimizerNotes.pdf\tabularnewline
 &  & \LaTeX Lab & \LaTeX{}.pdf; \LaTeX{}Lab.pdf\tabularnewline
5 & Thu 7/12 & Good programming practices & LecCluster.pdf\tabularnewline
 &  & Accessing cluster resources & ProgrammingPractices.pdf\tabularnewline
 &  & \emph{PS2 due by 11:59 p.m.} & \tabularnewline
6 & Fri 7/13 & Go over code for Problem Set 2 & \tabularnewline
7 & Tue 7/17 & Graphics in Matlab; & LectureGraphics.pdf\tabularnewline
 &  & Inference & LectureStandardErrors.pdf\tabularnewline
8 & Thu 7/19 & Exporting estimation results & LectureGraphics.pdf\tabularnewline
 &  & \emph{PS3 due by 11:59 p.m.} & ExportationLab.pdf\tabularnewline
9 & Fri 7/20 & Go over code for Problem Set 3 & \tabularnewline
\hline 
\end{tabular}
\par\end{center}
\end{document}
